\documentclass{sydeStyle}
\usepackage{amsmath}
\usepackage{amssymb}
\usepackage{subfig}
\usepackage{listings}
\DeclareGraphicsExtensions{.png}
\graphicspath{{./files/}}

\coursecode{352}
\prof{Alexander Wong}
\title{Lab 1 Report\\
Fundamentals of Image Processing}
\date{\today}

\author{Andrew Cameron, 20252410}
\authortwo{Simon Ruggier, 20165414}

\begin{document}

\maketitle

\section*{Introduction}
Image processing has a variety of applications, from automating complicated
processes such as tumour detection, to more general tasks, such as image
resizing, contrast enhancement, and image compression.
This report experiments with several image transformations that are relevant to
these areas, observing their effects and in some cases analyzing their
effectiveness.

\section*{Digital Zooming}
The digital zooming methods investigated here are nearest neighbour, bilinear interpolation, and bicubic interpolation.
Nearest neighbour produces blocky images, which is to be expected, as it simply copies nearby values exactly.
It works well in regions with very little change, because it assumes there is none.
If that assumption is incorrect, nearest neighbour will introduce error.
The interpolation techniques, on the other hand, produced smoother images; by interpolating unknown values, pixellated edges are replaced with smooth gradients.
They work better in regions of change, with bicubic interpolation working especially well in regions of fine detail.
As a higher-order interpolation algorithm, it is to be expected that it produces less error than bilinear interpolation.

The PSNR values for the different digital zooming algorithms' outputs are summarised in Table \ref{tab:psnr}.
Nearest neighbour, although producing a blocky output, performs well when it comes to PSNR.
This makes sense, as during the reduction phase, interpolation was used to downsample the image.
This means that an area of several pixels in the original image will be averaged to produce a single pixel in the downsampled image.
When this average is blown back up, then, it makes sense that it will have little error; it already contains information from the pixels it's supposed to imitate.
Bilinear interpolation, on the other hand, incurs error for smoothing out sharp details.
Anywhere there are sharp edges in the original will be blurred after bilinear resizing.
Bicubic interpolation doesn't perform as poorly as bilinear because of the way it weights closer pixels higher in its reconstruction.
Because of that, and because it uses farther pixels to incorporate the rate of change as well as the curvature, bicubic interpolation does a better job of estimating the lost pixels, and earns a higher PSNR for it.
Clearly, although PSNR gives some measure of the algorithms' quality, it doesn't account for the entirety of the perceived image quality.

Although the Lena image appears to be better after reconstruction than the Cameraman image, this could be attributed to a number of factors unrelated to the algorithms themselves.
For instance, the former is four times the resolution of the latter, and there is simply more detail present.
Furthermore, the \emph{rgbtogray} function performs some arbitrary tone mapping to convert from colour to grayscale, which may result in more information in Lena than Cameraman has to begin with.

\begin{table}
	\centering
	\begin{tabular}{| l | r | r |}
		\hline
		\multicolumn{3}{|c|}{PSNR (dB)} \\ \hline
		& Lena & Cameraman \\ \hline
		Nearest Neighbour & 32.752 & 31.314 \\ \hline
		Bilinear Interpolation & 32.746 & 30.978 \\ \hline
		Bicubic Interpolation & 33.156 & 31.271 \\
		\hline
	\end{tabular}
	\caption{Peak signal-to-noise for Lena and Cameraman}
	\label{tab:psnr}
\end{table}
%%%% simple table - PSNR (dB)
%%%% column 1: Lena
% Nearest Neighbour: 32.752
% Bilinear Interpolation: 32.746
% Bicubic Interpolation: 33.156
%%%% column 2: cameraman
% Nearest Neighbour: 31.314
% Bilinear Interpolation: 30.978
% Bicubic Interpolation: 31.271

\section*{Discrete Convolution for Image Processing}
 \begin{figure}
	\begin{center}
		\subfloat[Lena]{\label{fig:lena_orig}\includegraphics[width=0.2\textwidth]{lena_orig.png}}
		\subfloat[Lena$*h1$]{\label{fig:lena_conv_h1}\includegraphics[height=0.2\textwidth]{lena_conv_h1.png}}
		\subfloat[Lena$*h2$]{\label{fig:lena_conv_h2}\includegraphics[width=0.2\textwidth]{lena_conv_h2.png}}
		\subfloat[Lena$*h3$]{\label{fig:lena_conv_h3}\includegraphics[height=0.2\textwidth]{lena_conv_h3.png}}
		\subfloat[absolute value of Lena$*h3$]{\label{fig:lena_conv_h3_abs}\includegraphics[height=0.2\textwidth]{lena_conv_h3_abs.png}}
	\end{center}
	\caption{Lena, convoluted with various impulse functions}
	\label{SummingAmp2}
\end{figure}
In order to experiment with convolution in the context of image processing,
three impulse functions, $h1$, $h2$, and $h3$ were convolved with the standard
Lena test image:
\begin{align*}
	h1 & = \begin{bmatrix}\frac{1}{6} & \frac{1}{6} & \frac{1}{6} & \frac{1}{6}
		& \frac{1}{6} & \frac{1}{6}\end{bmatrix} \\
	h2 & = \begin{bmatrix}\frac{1}{6} & \frac{1}{6} & \frac{1}{6} & \frac{1}{6}
		& \frac{1}{6} & \frac{1}{6}\end{bmatrix}^T \\
	h3 & = \begin{bmatrix}-1 & 1\end{bmatrix}
\end{align*}

Because $h1$ is a horizontal vector, performing a 2D convolution with it should
incorporate information into each pixel from horizontal neighbours of that pixel,
resulting in a horizontal blur effect; figure \ref{fig:lena_conv_h1} confirms this.

The second impulse function, shown convoluted with Lena in figure
\ref{fig:lena_conv_h2}, is a transposed version of the first.  As such, it would
be expected that it would result in a vertical blurring effect, which it does.

The third impulse function results in a much different effect.
It is also 1 dimensional, acting in the horizontal direction, but it simply
subtracts the right neighbouring pixel from each pixel.
Thus, the brightness of each pixel is proportional to the difference between it
and its right neighbour.
This results in a crude edge magnifying transformation, which can be seen in
figure \ref{fig:lena_conv_h3}.
However, observe that if the input image range is $[0,1]$, the output image will
be $[-1,1]$.
As such, some of the output is negative, and by default, this gets saturated to
black when the output is displayed.
If each pixel in the output image is converted to its absolute value, the result
in figure \ref{fig:lena_conv_h3_abs} is obtained.

Given these results, it is clear that convolution can be a convenient way to
express operations that linearly transform pixels using information from their
neighbours.
Such operations are widely applicable to image processing.

\section*{Fourier Analysis}
% TODO: make this an ordered list
% 1. What can you say about the general distribution of energy in the Fourier spectra? Why?

% 2. What characteristics about the test image can you infer from the Fourier spectra?

% 3. How did the Fourier spectra change from the original image (before rotation)?

% 4. What conclusions and observations can be made about image characteristics based on the Fourier spectra of both original image and the rotated image?

% 5. Describe how the reconstructed image from the amplitude component look like? What image characteristics does the amplitude component capture?

% 6. Describe how the reconstructed image from the phase component look like? What image characteristics does the phase component capture?


\section*{Point Operations for Image Enhancement}
Various pixel operations can be used to enhance the contrast of an image.
To study this, a low contrast image of a tire was subjected to various per-pixel
operations.
In order to evaluate the effects of these operations, histograms for each output
image are compared with the original image.
Histograms are a plot of how many pixels have a given intensity value, with
intensity on the x-axis and number of pixels on the y-axis.
This is useful in evaluating contrast enhancement of various operations because
it provides a visual representation of the global contrast of an image.
 \begin{figure}
	\begin{center}
		\subfloat[Image]    {\includegraphics[height=0.42\textwidth]{tire_orig.png}}
		\subfloat[Histogram]{\includegraphics[height=0.42\textwidth]{tire_orig_hist.pdf}}
	\end{center}
	\caption{Tire image}
\end{figure}

The histogram for the original image shows that most of the pixels in the image
are concentrated in the low intensities, which results in low contrast, with
most of the image looking dark.

 \begin{figure}
	\begin{center}
		\subfloat[Image]    {\includegraphics[height=0.42\textwidth]{tire_negative.png}}
		\subfloat[Histogram]{\includegraphics[height=0.42\textwidth]{tire_negative_hist.pdf}}
	\end{center}
	\caption{Tire image after negative transform}
\end{figure}
As a first test, a negative transform is applied to the image.
This causes all dark pixels to become bright, and bright pixels to become dark.
Because most pixels in the input image were dark, the output image has mostly
bright pixels.
The histogram is flipped horizontally, and most pixels are within a small,
bright range of intensities, resulting in no improvement in contrast.

 \begin{figure}
	\begin{center}
		\subfloat[Image]    {\includegraphics[height=0.42\textwidth]{tire_gamma_13.png}}
		\subfloat[Histogram]{\includegraphics[height=0.42\textwidth]{tire_gamma_13_hist.pdf}}
	\end{center}
	\caption{Tire image after power-law transformation with $\gamma=1.3$}
\end{figure}
Next, power-law transformations are tested, in which each pixel is transformed
according to the equation:
\begin{displaymath}
	I_{out} = I_{in}^\gamma
\end{displaymath}
Two values for the gamma term $\gamma$ are tested, $1.3$ and $0.5$.
Because the range of values being transformed is between zero and one, a gamma
term that is less than one nonlinearly shifts the intensity distribution toward
brighter values, while a gamma term above one shifts the distribution to darker
values.
Because the image already has a very dark intensity distribution, adjusting with
a gamma term of 1.3 squashes the intensity distribution further, lowering
contrast and making the image look darker, with features that are harder to
distinguish in the low contrast.
 \begin{figure}
	\begin{center}
		\subfloat[Image]    {\includegraphics[height=0.42\textwidth]{tire_gamma_05.png}}
		\subfloat[Histogram]{\includegraphics[height=0.42\textwidth]{tire_gamma_05_hist.pdf}}
	\end{center}
	\caption{Tire image after power-law transformation with $\gamma=0.5$}
\end{figure}
Conversely a gamma term of 0.5 causes the intensity distribution to be brighter
overall, and spreads out the intensities of the dark pixels in the image (i.e.
most of them), thus improving contrast.
The resulting image is much clearer, and features are easier to distinguish.
Given the above points, the gamma term of 0.5 is a better choice for this image.
If the intensity distribution was highest in the bright intensities, or if
working with the negative transformed version of this image, the gamma term of
1.3 would have been a better choice for improving contrast.

 \begin{figure}
	\begin{center}
		\subfloat[Image]    {\includegraphics[height=0.42\textwidth]{tire_histeq_256.png}}
		\subfloat[Histogram]{\includegraphics[height=0.42\textwidth]{tire_histeq_256_hist.pdf}}
	\end{center}
	\caption{Tire image after histogram equalization}
\end{figure}
Finally, histogram equalization is performed on the image.
Although ideally, this equalization would produce a flat histogram, this is not
possible with discrete intensity values, because all pixels with the same
intensity are transformed equally.
Thus, spikes corresponding to above-average intensities on the histogram are
merely shifted along the x-axis.
However, it is possible to map multiple input intensity values to the same
output intensity, throwing out some information for the few pixels that are
spread out in the bright part of the intensity distribution.
Because most pixel intensities are densely distributed in the darker
intensities, this transformation causes a big increase in brightness for the
darker pixels in the image, which makes the output image look unrealistically
bright.

\section*{Conclusions}
Convolution of an image was demonstrated with 3 different impulse functions,
illustrating that convolution can have very distinct effects depending on the
chosen impulse function.

Various methods for improving contrast were performed and compared, using a
low-contrast image containing mostly dark pixels.
It was observed that a negative transform does nothing to improve the contrast,
because it doesn't shift pixel intensities relative to each other.
Power-law transformations can improve the contrast by non-linearly transforming
the intensity distribution to spread out pixels that have a small range of
intensities, so long as the exponent term is chosen so that the transformation occurs in
the right direction.
Histogram equalization automatically increases the contrast as much as possible,
but this can come at the expense of a realistic appearance for the image.

\section*{Appendix A: Source Code}
\lstset{
	language=Octave,
	% code font size
	basicstyle=\scriptsize,
	tabsize=2,
	breaklines=true,
	breakatwhitespace=false,        % sets if automatic breaks should only happen at whitespace
	% show filename as title
	title=\lstname,
}

\lstinputlisting{part3.m}
\lstinputlisting{part4.m}
\lstinputlisting{part5.m}
\lstinputlisting{part6.m}

\end{document}
