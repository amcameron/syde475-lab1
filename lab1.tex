\documentclass{sydeStyle}
\usepackage{amsmath}
\usepackage{amssymb}
\DeclareGraphicsExtensions{.png}

\coursecode{352}
\prof{Alexander Wong}
\title{Lab 1 Report\\
Fundamentals of Image Processing}
\date{\today}

\author{Andrew Cameron, 20252410}
\authortwo{Simon Ruggier, 20165414}

\begin{document}

\maketitle

\section*{Introduction}
Lorem ipsum dolor sit amet, consectetur adipiscing elit. Morbi quis cursus
sapien. Integer id ultrices augue. Donec quis arcu at odio dignissim
tincidunt a ac nibh. Sed condimentum tortor in odio auctor luctus.
Phasellus vitae aliquet nibh. Pellentesque felis lorem, rutrum euismod
cursus at, mattis imperdiet ipsum. Etiam eget auctor mauris. Vestibulum
mollis libero eget leo pretium nec eleifend tortor mattis. Aliquam aliquet,
tellus et hendrerit porttitor, magna ante congue justo, vulputate mollis
quam purus vitae felis. In hac habitasse platea dictumst.

\section*{Digital Zooming}
% TODO: make this an ordered list
% 1.
%All of the up-sampled images were clearly inferior to the original, although
%some were better than others.  Nearest neighbour is clearly the worst;
%the resulting image is very blocky and appears identical to the down-sampled
%image.  Bilinear interpolation is much better, although quite blurry compared
%to the original, whereas bicubic interpolation, although still blurry compared
%with the original, has clearer features than bilinear.

% 2. How do the different methods compare to each other in terms of PSNR as well as visual quality?  Why?
\begin{table}
	\centering
	\begin{tabular}{| l | r | r |}
		\hline
		\multicolumn{3}{|c|}{PSNR (dB)} \\ \hline
		& Lena & Cameraman \\ \hline
		Nearest Neighbour & 32.752 & 31.314 \\ \hline
		Bilinear Interpolation & 32.746 & 30.978 \\ \hline
		Bicubic Interpolation & 33.156 & 31.271 \\
		\hline
	\end{tabular}
	\caption{Peak signal-to-noise for Lena and Cameraman}
	\label{tab:psnr}
\end{table}
%%%% simple table - PSNR (dB)
%%%% column 1: Lena
% Nearest Neighbour: 32.752
% Bilinear Interpolation: 32.746
% Bicubic Interpolation: 33.156
%%%% column 2: cameraman
% Nearest Neighbour: 31.314
% Bilinear Interpolation: 30.978
% Bicubic Interpolation: 31.271
2. None of the up-sampled images are as good as the original images.  Nearest
neighbour upsampling produces blocky images, whereas bilinear and
bicubic interpolation both produce slightly better images, although both are
blurred.  Bicubic interpolation appears slightly less blurred than bilinear.
The PSNR values, as shown in Table \ref{tab:psnr}, are quite close for all
three methods; no algorithm comes out a clear winner, because all three
behave similarly: they upsample by using information from nearby pixels.

% 3. What parts of the image seems to work well using these digital zooming methods? What parts of the image doesn’t? Why?

% 4. Compare the zooming results between Lena and Cameraman. Which image results in higher PSNR?  Which image looks better when restored to the original resolution using digital zooming methods?  Why?

% 5. What does the PSNR tell you about each of the methods? Does it reflect what is observed visually?

\section*{Discrete Convolution for Image Processing}
Fusce at hendrerit est. Duis vel sem eget est vehicula facilisis. Sed
mattis laoreet nulla non suscipit. Mauris dui neque, sollicitudin vel
pharetra eget, rhoncus id justo. Suspendisse volutpat nisi eget sapien
mattis bibendum non ultricies libero. Maecenas bibendum cursus consequat.
Ut aliquet tortor ut lacus blandit auctor. Donec scelerisque, libero eu
eleifend mollis, tellus urna cursus enim, sit amet pretium dui libero vel
enim. Ut ultrices feugiat mi id interdum. Ut eu justo a sem varius
dignissim nec sit amet est. In a eros dolor, vel condimentum ante. Lorem
ipsum dolor sit amet, consectetur adipiscing elit. Praesent ornare tempor
blandit. Maecenas a nisi ac velit feugiat laoreet. Fusce felis nibh,
accumsan faucibus consectetur et, rhoncus vel orci. Pellentesque dapibus,
massa eu blandit convallis, lorem orci iaculis neque, vel elementum est
augue sit amet massa.

\section*{Fourier Analysis}
% TODO: make this an ordered list
% 1. What can you say about the general distribution of energy in the Fourier spectra? Why?

% 2. What characteristics about the test image can you infer from the Fourier spectra?

% 3. How did the Fourier spectra change from the original image (before rotation)?

% 4. What conclusions and observations can be made about image characteristics based on the Fourier spectra of both original image and the rotated image?

% 5. Describe how the reconstructed image from the amplitude component look like? What image characteristics does the amplitude component capture?

% 6. Describe how the reconstructed image from the phase component look like? What image characteristics does the phase component capture?


\section*{Point Operations for Image Enhancement}
In imperdiet ultrices congue. Nam ac mauris nec lectus vestibulum faucibus
porta dictum metus. Ut sem purus, tincidunt in scelerisque at, blandit non
nunc. Donec quis felis quam. Nulla facilisis gravida mattis. Phasellus non
augue lectus, eget venenatis justo. Aenean in nisl purus. Quisque nec felis
erat. Nunc vel nibh eget lectus blandit posuere non at augue. Donec sed
ante a justo mattis elementum. Aenean sollicitudin dignissim nibh, vitae
ultricies quam dictum et.

\section*{Conclusions}
Vivamus semper ipsum id nibh aliquam eu tincidunt tortor scelerisque. Class
aptent taciti sociosqu ad litora torquent per conubia nostra, per inceptos
himenaeos. Praesent ultricies lobortis purus ut porttitor. Nulla non eros
lorem, non pulvinar nisl. Lorem ipsum dolor sit amet, consectetur
adipiscing elit. Sed felis libero, hendrerit quis vulputate vitae, pulvinar
eu libero. Nulla facilisi. Proin facilisis justo in velit dictum placerat.
Curabitur mauris neque, vehicula quis dictum sed, sodales in diam. Cras
sodales sem at diam ultrices at sodales mi dapibus. Suspendisse quis neque
sit amet quam dapibus ornare at et dolor.

\section*{Appendix A: Source Code}
Ut placerat dolor vel lectus consectetur scelerisque. Maecenas facilisis
metus eget velit imperdiet dignissim. In porttitor hendrerit urna, eget
fermentum risus ornare sit amet. Nunc ullamcorper diam volutpat mauris
molestie tempus vitae tincidunt enim. Donec quis rhoncus arcu. Suspendisse
magna diam, euismod vel ultricies nec, ornare a metus. Class aptent taciti
sociosqu ad litora torquent per conubia nostra, per inceptos himenaeos. In
hac habitasse platea dictumst. Suspendisse pretium auctor turpis, lobortis
condimentum enim bibendum a. Donec libero ligula, tempor eu gravida eget,
vehicula vitae nisi. Class aptent taciti sociosqu ad litora torquent per
conubia nostra, per inceptos himenaeos.

\end{document}
